\documentclass{article}
\usepackage[numbered, framed]{mcode}
% Langue
\usepackage[utf8]{inputenc}
\usepackage[T1]{fontenc}      
\usepackage[francais]{babel}

% Mise en forme générale
\usepackage[top=2.5cm,bottom=2.5cm,right=2.5cm,left=2.5cm]{geometry}
\usepackage{subfigure}
% Package divers
\usepackage{chemist} 
\usepackage[version=3]{mhchem}
\usepackage{chemfig}
\usepackage[squaren, Gray]{SIunits}
\usepackage{sistyle}
\usepackage[autolanguage]{numprint}
\usepackage{url}
\usepackage{rotating}
\usepackage{xcolor,colortbl}
\definecolor{Gray}{gray}{0.85}

\usepackage{hyperref}
\hypersetup{
    colorlinks,
    citecolor=black,
    filecolor=black,
    linkcolor=black,
    urlcolor=black
}

% Nouvelles commandes
\newcommand{\std}{\ensuremath{^{\circ}}}
\newcommand\ph{\ensuremath{\mathrm{pH}}}
\newcommand{\annexe}{\part{Annexes}\appendix}
\newcommand{\biblio}[1]{\bibliographystyle{plain}\bibliography{#1}\nocite{*}}

\newcommand{\doctitle}[1]{
	\title{#1}
	\author{\textbf{Groupe 124.3}\\
	\textsc{Frenyo} Péter (6266-12-00)\\
	\textsc{Gillain} Nathan (7879-12-00)\\
	\textsc{Lamine} Guillaume (7109-13-00)\\
	\textsc{Piraux} Pauline (2520-13-00)\\
	\textsc{Paris} Antoine (3158-13-00)\\
	\textsc{Quiriny} Simon (4235-13-00)\\
	\textsc{Schrurs} Sébastien (7978-13-00)}
	\date{\today}

	\begin{document}

	\maketitle
	\tableofcontents
}
\doctitle{Projet 3 - Gestion des réactifs}

% ------------------------------------------
% INTRODUCTION
% ------------------------------------------
\section{Introduction}
Dans le cadre du projet P3, nous avons réalisé une 
étude théorique sur la fabrication
industrielle d'ammoniac.

Ce rapport fournit premièrement les informations relatives
aux sources de réactifs utilisés,
aux réactifs nécessaires à la production d'une quantité fixée 
d'ammoniac, de la quantité d'eau nécessaire au refroidissement
permettant d'atteindre la température requise pour que la 
réaction Haber-Bosch ait lieu ainsi que l'énergie nécessaire au 
réformage primaire.

Nous fournissons ensuite un outil de gestion écrit en MATLAB 
synthétisant les calculs des quantités de matières nécessaires 
à la production d'une masse fixe d'ammoniac à une certaine 
température du réacteur du réformage primaire. Cet outil fournit 
les valeurs en moles par seconde de tous les composés à chaque étape de production.

Nous réalisons ensuite une étude paramétrique permettant la combinaison
optimale de température et de production quotidienne 
d'ammoniac correspondant à nos produits disponibles, et nous finissons
par un calcul du nombre de tubes nécessaires à un 
certain débit de \ce{CH_4} et de \ce{H_2O} dans le réformeur primaire.

% ------------------------------------------
% SOURCES DES REACTIFS
% ------------------------------------------
\section{Sources des réactifs}
	\subsection{Sources de diazote}
	\subsubsection{Procédé cryogénique}
	Ce procédé se base sur la séparation des différents constituants de 
	l'air en fonction de leur température 
	d'ébullition (l'oxygène \chemform{O_2} se condense avant le diazote \chemform{N_2}).
	L'air est purifié jusqu'à liquéfaction et les différents constituants 
	sont séparés dans une colonne de 
	rectification par distillation fractionnée. Cette méthode permet d'avoir
	du diazote \chemform{N_2} pur 
	à \numprint{99,99}\%. Cette méthode est efficace pour une consommation
	au-delà de $\unit{200}{\meter\cubed\per\hour}$ \cite{scf}. 

	\subsubsection{Perméation gazeuse}
	Ce procédé utilise les différentes vitesses d'effusion des molécules 
	de gaz à travers une membrane. 
	L'\chemform{O_2}, \chemform{H_2O} et le \chemform{CO_2} s'effusent 
	plus rapidement que le \chemform{N_2}.
	Cette méthode nous permet d'obtenir du \chemform{N_2} sec pur à 
	\numprint{95}-\numprint{99}. Ce procédé s'utilise pour des 
	débits forts variables ($\unit{3-1000}{\meter\cubed\per\hour}$) \cite{scf}.

	\subsubsection{Méthode de Ramsay}
	
	\begin{chemmath}
			NaNO_2(aq) + NH_4Cl(aq) \longrightarrow NaCl(aq) + 2H_2O(l) + N_2(g)
	\end{chemmath}
	
	On chauffe le mélange de \chemform{NaNO_2} et \chemform{NH_4Cl} pour 
	obtenir le \chemform{N_2} sous forme gazeuse.
	Un désavantage de cette méthode par rapport aux 2 premières est qu'il 
	faut acheter les réactifs. De plus, il faut utiliser de l'énergie pour 
	chauffer la réaction \cite{wiki-n2}.

	\subsection{Sources de dihydrogène}
		\subsubsection{Vaporeformage de méthane}
		2 réactions sont utilisées pour ce procédé \cite{afhypac} :
		
		\begin{chemmath} 
			CH_4 + H_2O \longleftrightarrow CO + 3H_2
		\end{chemmath}
		
		\begin{chemmath}
			CO + H_2O \longleftrightarrow CO_2 + H_2
		\end{chemmath}
		
		L'équation bilan obtenue est la suivante :
		
		\begin{chemmath}
			CH_4 + 2H_2O \longleftrightarrow CO_2 + 4 H_2
		\end{chemmath}
		
			Cette réaction nécessite un catalyseur : le nickel. Le rendement varie
			entre $40-45\%$. Le problème de cette 
			méthode est qu'elle rejette une grande quantité de \chemform{CO_2}, 
			gaz à effet de serre \cite{wiki-h2}.
			
		\subsubsection{Oxydation partielle d'hydrocarbure}
		\begin{chemmath}
			C_nH_m + \frac{n}{2} O_2 + \frac{3,76n}{2} N_2 \longrightarrow \frac{m}{2} H_2 + n CO + \frac{3,76n}{2} N_2
		\end{chemmath}
		L'air est comburant pour cette réaction, qui a besoin d'être catalysée.
		Son caractère exothermique aide à la catalyse. 
		L'inconvénient de cette méthode est son faible rendement \cite{wiki-h2}.

		\subsubsection{Electrolyse}
		Réaction à l'anode : 
		
		\begin{chemmath}
			2H_2O(l) \longrightarrow O_2(g) + 4 H^+(aq) + 4e^-
		\end{chemmath}
		
		Réaction à la cathode :
		
		\begin{chemmath}
			4H_2O(l) + 4e^- \longrightarrow 2H_2O(g) + 4OH^-(aq)
		\end{chemmath}
		
		L'équation bilan obtenue est la suivante :
		
		\begin{chemmath}
			2H_2O(l) \longrightarrow 2H_2(g) + O_2(g)
		\end{chemmath}
		
	La réaction nécessite une grande quantité d'électricité. L'eau, quant à elle, 
	est présente en quantité illimitée 
	et est peu coûteuse. En pratique, cette méthode est très peu utilisée \cite{wiki-h2}.

% ------------------------------------------
% CALCUL DU FLUX DES REACTIFS
% ------------------------------------------
\section{Calcul du flux des réactifs}
\paragraph{Hypothèse}
Lors du calcul de ces flux, nous avons utilisé l'hypothèse que 
tous les réactifs sont 
consommés par le réacteur.

\paragraph{Calculs}
L'équation de la réaction de production de l'ammoniac par le
procédé \textsc{Haber-Bosch} est donnée par :

	\begin{chemmath}
			\frac{1}{2}N_2(g) + \frac{3}{2}H_2(g) \longrightarrow NH_3(g) + \Delta H
 	\end{chemmath}
	
En sachant que l'on cherche à produire \unit{1000}{\ton} de \chemform{NH_3} par jour, 
on calcule assez facilement le flux de \chemform{N_2} (les détails de calculs ont été omis
dans ce rapport)

	$$m_{N_2} = \unit{823.5}{\ton\per\dday}$$

et le flux de \chemform{H_2}

	$$m_{H_2} = \unit{176.4}{\ton\per\dday}.$$

% ------------------------------------------
% CALCUL DU DEBIT D'EAU NECESSAIRE POUR...
% ------------------------------------------
\section{Calcul du débit d'eau nécessaire pour refroidir le réactif}
\paragraph{Hypothèses}
\begin{itemize}
	\item Les capacités calorifiques ne dépendent pas de la température ;
	\item La pression dans le réacteur est constante et vaut \unit{10^5}{\pascal}.
\end{itemize}

\paragraph{Première méthode : on considère les capacités calorifiques constantes}
Pour la réaction donnée dans la section précédente, on a $\Delta H(\unit{298.15}{\kelvin})
 = \unit{-46 \cdot 10^3}{\joule}$ pour une mole de \chemform{NH_3(g)} produite \cite{atkins}.
Comme la réaction a lieu à \unit{500}{\celsius} (c'est à dire \unit{773.15}{\kelvin}), il
va falloir calculer $\Delta H(\unit{773.15}{\kelvin})$. Pour cela, nous avons besoin
des capacités calorifiques moyenne à pression constante de chacun des réactifs et des produits
de la réaction. Nous trouvons ces données dans une table \cite{atkins}.

	$$
	\left\{
		\begin{array}{rl}
			C_{p_{NH_3(g)}} &= \unit{37}{\joule\per\mole\kelvin}\\
			C_{p_{H_2(g)}} 	&= \unit{28.836}{\joule\per\mole\kelvin}\\
			C_{p_{N_2(g)}} 	&= \unit{29.124}{\joule\per\mole\kelvin}
		\end{array}
	\right.
	$$

On a donc :

$$\Delta H(\unit{773.15}{\kelvin}) = \Delta H(\unit{298.15}{\kelvin})
+ \int_{298.15}^{773.15} C_{p_{NH_3(g)}} dT - \frac{1}{2}\int_{298.15}^{773.15} C_{p_{N_2(g)}} dT
- \frac{3}{2} \int_{298.15}^{773.15} C_{p_{H_2(g)}} dT = \unit{-5.5887 \cdot 10^4}{\joule\per\mole}$$

Pour la quantité de \chemform{NH_3} à produire par jour 
(à savoir $\unit{58.83 \cdot 10^6}{\mole}$),
la quantité de chaleur produite est donc :

$$q = \Delta H(\unit{773.15}{\kelvin}) \cdot n_{NH_3} = \unit{-3.28786 \cdot 10^{12}}{\joule}$$

En connaissant la capacité calorifique de l'eau $C_{H_2O(g)} =
\unit{4185.5}{\joule\per\kilo\gram\kelvin}$ ~\cite{atkins}et en égalant
$q$ à $m_{H_2O} \cdot C_{H_2O(g)} \cdot \Delta T$ avec 
$\Delta T = 90 - 25 = \unit{65}{\kelvin}$, on trouve un
flux d'eau égal à

$$m_{H_2O} = \unit{1.2085 \cdot 10^7}{\kilo\gram\per\dday} \Rightarrow V_{H_2O} \approx \unit{1.2085 \cdot 10^7}{\liter\per\dday} 
= \unit{139.8}{\liter\per\second}$$

\paragraph{Deuxième méthode : les capacités calorifiques dépendent de la température}
On peut être plus précis en utilisant les capacités calorifiques suivantes \cite{hc-table} :

	$$
	\left\{
		\begin{array}{rl}
			C_{p_{NH_3(g)}}(T) &= \unit{31.81 + (15.48 \cdot 10^{-3})T + (5.86 \cdot 10^{-6})T^2}{\joule\per\mole\kelvin}\\
			C_{p_{H_2(g)}}(T) 	&= \unit{29.30 - (0.84 \cdot 10^{-3})T + (2.09 \cdot 10^{-6})T^2}{\joule\per\mole\kelvin}\\
			C_{p_{N_2(g)}}(T) 	&= \unit{27.62 + (4.19 \cdot 10^{-3})T}{\joule\per\mole\kelvin}
		\end{array}
	\right.
	$$

En refaisant le calcul ci-dessus en tenant compte de la variation des capacités
calorifiques en fonction
de la température, nous obtenons un débit un peu inférieur 
de \unit{135.662}{\liter\per\second}. On remarque
que l'erreur faite en utilisant l'approximation de la section
précédente est de 3\%.


% ------------------------------------------
% BILAN DE MATIERE
% ------------------------------------------
\section{Bilan de matiere}
Pour la compréhension de cette section, nous vous renvoyons vers le flow-sheet,
dans l'annexe \ref{appendix:flow-sheet}. On pose le flux de \chemform{NH_3(g)} 
à la sortie égal à $\unit{m_{NH_3}}{\gram\per\dday} \footnote{Avec \unit{\gram\per\dday} = grammes par jour.}$. 

Nous utilisons une méthode bottom-up. Nous partons donc de la réaction : 
\begin{chemmath}
		\frac{1}{2}N_2(g) + \frac{3}{2}H_2(g) \longrightarrow NH_3(g) 
\end{chemmath}

Et ainsi,
 
$$n_{NH_3} = \unit{\frac{m_{NH_3}}{17}}{\mole\per\dday}$$

Ce qui donne les valeurs suivantes pour le nombre de moles final de $N_2$ et de $H_2$ nécessaire ("final" étant donné que, pour rappel, nous partons de la fin) : 

$$n_{N_2,f} = \frac{n_{NH_3}}{2} = \unit{\frac{m_{NH_3}}{34}}{\mole\per\dday}$$ 

et 

$$n_{H_2,f} = \frac{3}{2} \cdot n_{NH_3}$$

On sait que le \chemform{N_2(g)} provient uniquement de l'air entrant 
dans le réacteur du  réformage secondaire. Et prenant l'hypothèse
que l'air est composé de 78\% de \chemform{N_2(g)}, 21\% de \chemform{O_2(g)}
et 1\% d'\chemform{Ar(g)}, on peut déduire que $n_{air}$ entrant dans le 
réacteur du réformage primaire vaut : 

$$n_{air}= n_{N_2} \cdot \frac{100}{78} = \unit{\frac{m_{NH_3}}{26.52}}{\mole\per\dday}$$ 

Et :

$$n_{O_2}= n_{air} \cdot \frac{21}{100} = \unit{\frac{0.21m_{NH_3}}{26.52}}{\mole\per\dday}$$
$$n_{Ar}= n_{air} \cdot \frac{1}{100} = \unit{\frac{0.01m_{NH_3}}{26.52}}{\mole\per\dday}$$

La réaction dans le réformage secondaire, durant laquelle l'air est inséré, est la suivante :
\begin{chemmath}
	2CH_4(g) + O_2(g) \Longrightarrow 2CO(g) + 4 H_2(g)
\end{chemmath}  

Nous savons par hypothèse que le
\chemform{CH_4(g)} et le \chemform{O_2(g)} sont présents en quantité stoechiométrique. 
Nous avons donc la quantité suivante de \chemform{CH_4} à l'entrée du réformage secondaire (RS) : 

$$n_{CH_4,RS} = 2 \cdot n_{O_2} = \unit{\frac{0.42m_{NH_3}}{26.52}}{\mole\per\dday}$$

Ainsi que les quantités suivantes de \chemform{CO} et de \chemform{H_2} créés 

$$n_{CO,RS} = n_{CH_4,RS} =  \unit{\frac{0.42m_{NH_3}}{26.52}} {\mole\per\dday}$$
$$n_{H_2,RS} = 2 \cdot n_{CH_4,RS} =  \unit{\frac{0.84m_{NH_3}}{26.52}}{\mole\per\dday}$$

On s'intéresse ensuite à la réaction du Water-Gas-Shift : 

\begin{chemmath}
	CO(g) + H_2O(g) \Longrightarrow CO_2(g) + H_2(g)
\end{chemmath} 

On sait que $n_{CO,WGS} = n_{CO,RP} + n_{CO,RS}$ avec $n_{CO,WGS}$ 
le nombre de moles de \ce{CO} utilisées dans la réaction Water-Gas-Shift,
$n_{CO,RP}$ celui a la sortie du réformage primaire et $n_{CO,RS}$ celui 
à la sortie du réformage secondaire.

Dans la réaction Water-Gas-Shift, nous avons $n_{CO,WGS}$ moles de \ce{CO},
un excès de \ce{H_2O} en réactifs, et $n_{CO,WGS}$ moles de \ce{CO_2} et de 
\ce{H_2} formés :

$$n_{CO_2} = n_{CO} = n_{H_2}.$$

S'il reste du \chemform{H_2O} à la fin de cette réaction, le nombre de moles d'eau
à la sortie de la réaction Water-Gas-Shift sera égal aux nombre de moles d'eau à la
sortie du réformage primaire moins $n_{CO,WGS}$ (le nombre de moles utilisés dans 
la réaction Water-Gas-Shift) :

$$n_{H_2O,\text{dégagé}} = n_{{H_2O,RP}}- n_{CO,WGS}.$$

On peut aussi déduire que le nombre de moles de \ce{CO_2} dégagé à la fin de la 
réaction Water-Gas-Shift, $n_{CO_2-tot}$, vaut la somme du nombre de moles de \ce{CO_2}
produits par le réformage primaire et par la réaction Water-Gas-Shift : 

$$n_{CO_2-\text{dégagé}} = n_{{CO_2,RP}} + n_{{CO_2,WGS}}.$$

Etant donné que l'on a le nombre de moles de \ce{H_2} nécessaire à la production de
$\unit{m_{NH_3}}{\gram\per\dday}$, qui est de $n_{H_2,f}$, et que l'on connait le 
nombre de moles de \ce{H_2} produit à travers les diverses réactions, nous savons 
les égaliser de la manière suivante (les indices \textit{RP}, \textit{RS} et \textit{WGS}
signifiant \textit{à la sortie} du réformage primaire, secondaire et de la réaction Water-Gas-Shift) : 

$$n_{H_2,f} = n_{H_2,RP} + n_{H_2,RS} + n_{H_2,WGS}$$

% ------------------------------------------
% EQUILIBRE DU REFORMAGE PRIMAIRE
% ------------------------------------------
\section{Equilibre du reformage primaire}
\subsection{Calcul de la constante d'équilibre}
Pour calculer la constante d'équilibre nous allons utiliser $K= \exp{\frac{-\Delta G}{RT}}$ avec $\Delta G = \Delta H - T\Delta S$.

\subsubsection{Première réaction}
Calculons la constante d'équilibre $K_1$ de la réaction suivante :

\begin{chemmath} 
 CH_4(g) + H_2O(g) \Leftrightarrow CO(g) + 3H_2(g).
\end{chemmath} 

Connaissant l'enthalpie en conditions standards \cite{atkins} et les capacités calorifiques dépendant de la température \cite{hc-table},
$$
\left\{
	\begin{array}{rl}
		C_{p_{CO}}(T) 		&= \unit{27.62 +(5.02 \cdot 10^{-3})T}{\joule\per\mole\kelvin} \\
		C_{p_{H_2}}(T) 		&= \unit{29.3-(0.84 \cdot 10^{-3})T + (2.09\cdot 10^{-6})T^2}{\joule\per\mole\kelvin} \\
		C_{p_{CH_4}}(T) 	&= \unit{14.23+(75.3 \cdot 10^{-3})T - (18\cdot 10^{-6})T^2}{\joule\per\mole\kelvin} \\
		C_{p_{H_2O}}(T) 	&= \unit{30.13+(10.46 \cdot 10^{-3})T}{\joule\per\mole\kelvin} 
	\end{array}
\right.
$$

on peut écrire

$$\Delta C_p(T) = 3C_{p_{H_2}}(T) + C_{p_{CO}}(T) - C_{p_{CH_{4}}}(T) - C_{p_{H_2O}}(T).$$

On peut donc directement calculer $\Delta H_1(T)$ :

$$
	\begin{array}{rl}
		 	 \Delta H_1(T)	= \Delta H(\unit{298.15}{\kelvin}) + \int_{298.15}^T \Delta C_p(T) dT 
											= \unit{188369.87 + 71.16 T -0.04163 T^2 + (8.09\cdot 10^{-6}) T^3}{\joule\per\mole} 
	\end{array}
$$	

Connaissant l'entropie en conditions standards \cite{atkins}, on peut 
également directement calculer $\Delta S_1$
 
$$
	\begin{array}{rl}
		 	 \Delta S_1(T)	&=  \Delta S(\unit{298.15}{\kelvin}) + \int_{298.15}^{T} \frac{\Delta C_p}{T}dT \\
											&= \unit{-167.05 + 71.16 \ln T -0.08326 T + (1.2135\cdot 10^{-5}) T^2}{\joule\per\mole\per\kelvin}
	\end{array}
$$	

On peut alors calculer $\Delta G_1$
 
 \begin{equation}
	\Delta G_1(T) = \unit{188369.9 - (71.16\ln T)T + 238.21T + 0.04163 T^2 -(4.045\cdot 10^{-6})T^3}{\joule\per\mole}
	\label{delta-g1}
 \end{equation} 

et donc enfin obtenir $K_1$

$$K_1 = \exp{\frac{-\Delta G_1}{RT}}$$

où $\Delta G_1$ est donnée par l'équation \ref{delta-g1}.

\subsubsection{Deuxième réaction}
Calculons maintenant la constante d'équilibre $K_2$ de la deuxième réaction :

\begin{chemmath} 
	CO(g) + H_2O(g) \Leftrightarrow CO_2(g) + H_2(g)
\end{chemmath} 

Connaissant l'enthalpie en conditions standards \cite{atkins} et les capacités calorifiques dépendant de la température \cite{hc-table},

$$
	\begin{array}{rl}
		C_{p_{CO_2}}(T)=32.22 +(22.18 \cdot 10^{-3})T - (3.35 \cdot 10^{-6})T^2\\
	\end{array}
$$

on peut écrire

$$\Delta C_p(T) = C_{p_{H_2}}(T) + C_{p_{CO_2}}(T) - C_{p_{CO}}(T) - C_{p_{H_2O}}(T).$$

On peut donc directement calculer $\Delta H_2(T)$ 

$$
	\begin{array}{rl}
		 \Delta H_2(T)	&=  \Delta H(\unit{298.15}{\kelvin}) + \int_{298.15}^{T} \Delta C_p(T) dT \\
										&=  \unit{-42533.33+3.77T+(2.93\cdot 10^-3)T^2-(4.2\cdot 10^-7)T^3}{\joule\per\mole}.
	\end{array}
$$	

Connaissant l'entropie en conditions standards \cite{atkins}, on peut
également directement calculer $\Delta S_2(T)$

$$
	\begin{array}{rl}
		 	\Delta S_2(T)	&= \Delta S(\unit{298.15}{\kelvin}) 
											 + \int_{298.15}^{T} \frac{\Delta C_p(T)}{T}dT \\
										&= \unit{-65.9 + 3.77 \ln(T) +(5.86\cdot 10^{-3})T -(6.3 \cdot 10^{-7})T^2}{\joule\per\mole\kelvin}
	\end{array}
$$	

Nous pouvons donc calculer $\Delta G_2$
 
 \begin{equation}
	\Delta G_2= \unit{-42533.33 - (3.77 \ln(T))T +69.67 T -(2.93 \cdot 10^{-3})T^2 + (2.1\cdot 10^{-7})T^3}{\joule\per\mole} 
	\label{delta-g2}
 \end{equation}
 
pour enfin obtenir $K_2$

	$$K_2 = \exp{\frac{-\Delta G_2}{RT}}$$

où $\Delta G_2$ est donné par l'équation \ref{delta-g2}.

\paragraph{Remarque} Pour plus de précisions et afin d'automatiser le calcul
de ces constantes d'équilibres, nous avons créé deux fonctions Matlab, \lstinline{ComputeK1(T)}
et \lstinline{ComputeK2(T)} qui suivent exactement cette démarche.

\subsection{Etat d'avancement}
Analysons maintenant plus en détails les deux réactions qui ont lieu dans le réformage primaire.
Ces deux réactions se passent à l'équilibre dans le même réacteur. 
De plus, dans la gamme de températures qui nous intéressent, on considère tous les composantes à l'état gazeux.
Nous ferons l'hypothèse que ces gaz se comportent comme des gaz parfaits.
Voici donc les tableaux d'avancement en conséquence, dans les tables \ref{avancement1} et \ref{avancement2}.
  
	\begin{table}[ht!]
		\centering
		\begin{tabular}{c|cccccccc}
									& \ce{CH_4(g)} 				&+& \ce{H_2O(g)} 			 	&	$\Leftrightarrow$ 		& \ce{CO(g)} 			&+& \ce{3H_2(g)} \\
			\hline
			$n_i$ 			& $n_{01}$ 						& & $n_{02}$						& 											& 0								&	& 0 \\
			$n_{eq}(x)$	&	$n_{01}-x$ 					& & $n_{02}-x-y$				& 											& $x-y$ 					&	& $3x+y$ \\
			\hline 
			$a_{eq}$		& $\frac{n_{01}-x}{n_{gaz,tot}}\frac{p}{p\std}$ &
																				& $\frac{n_{02}-x-y}{n_{gaz,tot}}\frac{p}{p\std}$ &
																															& $\frac{x-y}{n_{gaz,tot}}\frac{p}{p\std}$ &
																																									& $\frac{3x+y}{n_{gaz,tot}}\frac{p}{p\std}$
		\end{tabular}
		\caption{Tableau d'avancement de la première réaction.}
		\label{avancement1}
	\end{table}
	
	\begin{table}[ht!]
		\centering
		\begin{tabular}{c|cccccccc}
									& \ce{CO(g)} 				&+& \ce{H_2O(g)} 			 		&	$\Leftrightarrow$ 		& \ce{CO_2(g)} 			&+& \ce{H_2(g)} \\
			\hline
			$n_i$ 			& $x$ 							& & $n_{02}-x$						& 											& 0								&	& $3x$ \\
			$n_{eq}(x)$	&	$x-y$ 						& & $n_{02}-x-y$					& 											& $y$ 						&	& $3x+y$ \\
			\hline 
			$a_{eq}$		& $\frac{x-y}{n_{gaz,tot}}\frac{p}{p\std}$ &
																				& $\frac{n_{02}-x-y}{n_{gaz,tot}}\frac{p}{p\std}$ &
																															& $\frac{y}{n_{gaz,tot}}\frac{p}{p\std}$ &
																																									& $\frac{3x+y}{n_{gaz,tot}}\frac{p}{p\std}$
		\end{tabular}
		\caption{Tableau d'avancement de la deuxième réaction.}
		\label{avancement2}
	\end{table}
	
Ici, $x$ et $y$ sont respectivement les avancements des première et deuxième réactions. Nous travaillons en moles par jour.
On remarque bien que les quantités à l'équilibre correspondent dans les deux tableaux.
Avant de s'attaquer à l'écriture des quotients réactionnels à l'équilibre, notons que le nombre total de moles de gaz
se trouve en prenant une seule fois le nombre de moles de chaque composant.
    
On a donc : $n_{gaz,tot} = n_{01} + n_{02} + 2x$ 
  
Nous pouvons écrire nos deux équilibres : 
 
$$
	\left\{
		\begin{array}{rl}
			K_1 =& \frac{(x-y)(3x+y)^3p_{tot}^2}{(n_{02}-x)(n_{02}-x-y)n_{gaz,tot}^2{p\std}^2} \\
			K_2 =& \frac{y(3x+y)}{(x-y)(n_{02}-x-y)}
		\end{array}
	\right.
$$

Pour ces équations, $p_{tot}$ est la pression à la sortie du réacteur, c'est-à-dire \unit{28}{\bbar}
et $p\std$ est la pression standard, c'est à dire \unit{1}{\bbar}.
$K_1$ et $K_2$ ont été calculés plus tôt en fonction de la température. 
De plus, grâce au bilan de matière fait précédement, nous obtenons les deux équations suivantes :

$$
	\left\{
		\begin{array}{rl}
			n_{01} - x =& \frac{0.42}{26.52} \cdot m_{NH_3} \\
			4x + 3\frac{1.26}{26.52}\cdot m_{NH_3} =& \frac{3a}{34}m_{NH_3}
		\end{array}
	\right.
$$
 
Ainsi, nous avons un système de quatre équations à quatre inconnues qui nous permettra d'exprimer
toutes les entrées/sorties en fonction de la température et du débit de \chemform{NH_3}. 
   
% ------------------------------------------
% BILAN D'ENERGIE
% ------------------------------------------	
\section{Bilan d'energie}
Afin d'évaluer la quantité totale d'énergie dont nous avons besoin pour mener à 
bien la synthèse d'ammoniac, nous allons regarder les quantités requises à chaque 
étape du processus pour ensuite déterminer la totalité des besoins énergétiques du
système.

Comme chaque réaction se passe à des températures différentes de la température ambiante,
nous calculerons le $\Delta H_{reaction}$ à température ambiante selon l'équation suivante 
$$\Delta H_{reaction} = \Sigma \Delta H_{f,produits} - \Sigma \Delta H_{f,reactifs}.$$

Ensuite, à l'aide des $C_{p}$ variables en fonction de la température, nous serons alors en 
mesure de déterminer le $\Delta H_{reaction}$ à la température voulue selon l'équation

$$\Delta H(T_2) = \Delta H(T_{1}) 
+ \int_{T_2}^{T_1} C_{p_{reactifs}} dT + \int_{T_1}^{T_2} C_{p_{produits}} dT$$ 

où $T_1$ est ici la température ambiante, soit \unit{298.15}{\kelvin}.

\paragraph{Réaction de combustion du four}
La combustion du méthane se passe dans le four, et fournit la totalité de l'énergie 
requise par l'ensemble du processus, avec un rendement de 75 pourcents.
La réaction se produit selon l'équation chimique suivante :

\begin{chemmath}
	CH_4 + 2O_2 \Longrightarrow CO_2 + 2 H_2O
\end{chemmath}

$$
	\begin{array}{rl}
	\Delta H_{reaction}		&=  \Sigma \Delta H_{f,produits} - \Sigma \Delta H_{f,reactifs} \\
												&=  (-393.51) + 2\cdot(-241.82) - (-74.81) + 2\cdot0 \\
												&=  (-877.15) - (-74.81)\\
												&=  \unit{-802.34}{\kilo\joule\per\mole}
	\end{array}
$$

La réaction se passe généralement à une température avoisinant les $\unit{1300}{\kelvin}$.
Voici les $C_p$ variables en fonction de la température des différents composants\cite{hc-table} 
(exprimés en \unit{\joule\per\mole\kelvin}) :

$$
	\left\{
		\begin{array}{rl}
			C_{p_{CH_4}}(T) 	&= 14.23 + 75.3\cdot10^{-3}T + (-18\cdot10^{-6})T^2 \\
			C_{p_{O_2}}(T) 		&= 25.73 + 12.97\cdot10^{-3}T + (-3.77\cdot10^{-6})T^2 \\
			C_{p_{CO_2}}(T) 	&= 32.22 + 22.18\cdot10^{-3}T + (-3.35\cdot10^{-6})T^2 \\
			C_{p_{H_2O}}(T) 	&= 30.13 + 10.46\cdot10^{-3}T 
		\end{array}
	\right.
$$	

Calculons maintenant le $\Delta H$ pour une température $T_2$ de $\unit{1300}{\kelvin}$ :

$$
	\begin{array}{rl}
		 	\Delta H(\unit{1300}{\kelvin}) 	&=  \Delta H(\unit{298.15}{\kelvin}) + \int_{1300}^{298.15} C_{p_{reactifs}} dT + \int_{298.15}^{1300} C_{p_{produits}} dT \\
																			&=  \unit{-805.99}{\kilo\joule\per\mole}
	\end{array}
$$	

On peut donc voir que cette réaction est largement exothermique, c'est elle qui fournira l'énergie nécessaire aux réactions du réformage primaire. En notant $\Delta H_1$ et $\Delta H_2$ les énergies nécessaires aux deux réactions
se produisant dans le reformage primaire, nous pouvons simplement calculer le nombre de mole de \ce{CH_4} à
l'entrée du four avec l'équation suivante

$$n_1\Delta H_1 + n_2\Delta H_2 = - 0.75\cdot\Delta H_{\text{four}}\cdot n_{CH_4}$$

où $n_1$ et $n_2$ sont les nombres de moles des réactifs réagissant dans les deux réactions
du reformage primaire (respectivement). Autrement dit, on a 

$$n_1 = n_{\ce{CH_4}_{\text{in}}} - n_{\ce{CH_4}_{\text{out}}}$$

et 

$$n_2 = n_{\ce{CO_2}_{\text{out}}}.$$

On peut vérifier cela sur le flow-sheet en annexe \ref{app:flow-sheet}.
C'est cette formule qui permet à l'outil de gestion de calculer les quantités
de réactifs du four.

\paragraph{Reformage primaire}
Dans notre travail, la température $T$ du réformage primaire est un paramètre. Nous obtiendrons donc une enthalpie 
dépendant de la température. 
Le réformage primaire est composé de deux équations.

La première réaction est donnée par 
\begin{chemmath} 
 CH_4(g) + H_{2}O(g) \Leftrightarrow CO(g) + 3H_2
\end{chemmath} 

Connaissant l'enthalpie en conditions standards \cite{atkins} et les capacités calorifiques dépendant de la température \cite{hc-table}:
$$
\left\{
	\begin{array}{rl}
		C_{p_{CO}}(T) 			&= \unit{27.62 +(5.02 \cdot 10^{-3})T}{\joule\per\mole\kelvin} \\
		C_{p_{H_2}}(T) 		&= \unit{29.3-(0.84 \cdot 10^{-3})T + (2.09\cdot 10^{-6})T^2}{\joule\per\mole\kelvin} \\
		C_{p_{CH_4}}(T) 	&= \unit{14.23+(75.3 \cdot 10^{-3})T - (18\cdot 10^{-6})T^2}{\joule\per\mole\kelvin} \\
		C_{p_{H_2O}}(T) 	&= \unit{30.13+(10.46 \cdot 10^{-3})T}{\joule\per\mole\kelvin} 
	\end{array}
\right.
$$

$$
	\begin{array}{rl}
		 	\Delta H_1(T) &= \Delta H(\unit{298.15}{\kelvin})  
										+ \int_{298.15}^{T} C_{p_{CO(g)}} dT + 3\int_{298.15}^{T} C_{p_{H_2(g)}} dT 
											+  \int_{T}^{298.15} C_{p_{CH_4(g)}} dT + \int_{T}^{298.15}C_{p_{H_{2}O_{(g)}}}dT\\
										&=  \unit{188369.87 + 71.16 T -0.04163 T^2 + (8.09\cdot 10^{-6}) T^3}{\joule\per\mole} % Unité correcte?
	\end{array}
$$	

La deuxième réaction est donnée par 
\begin{chemmath} 
	CO(g) + H_{2}O(g) \Leftrightarrow CO_2(g) + H_2
\end{chemmath} 

Connaissant l'enthalpie en conditions standards \cite{atkins} et les capacités calorifiques dépendant de la température \cite{hc-table}:

$$
\begin{array}{rl}
C_{p_{CO_2}}(T) = \unit{32.22 +(22.18 \cdot 10^{-3})T + (3.35 \cdot 10^{-6})T^2}{\joule\per\mole\kelvin} \\
\end{array}
$$

$$
	\begin{array}{rl}
		  \Delta H_2(T)	&=   \Delta H(\unit{298.15}{\kelvin}) 
											 + \int_{298.15}^{T} C_{p_{CO_2(g)}} dT + \int_{298.15}^{T} C_{p_{H_2(g)}} dT 
											 +  \int_{T}^{298.15} C_{p_{CO(g)}} dT + \int_{T}^{298.15}C_{p_{H_{2}O_{(g)}}}dT \\
										&=  \unit{-42533.33+3.77T+(2.93\cdot 10^{-3})T^2-(4.2\cdot 10^-7)T^3}{\joule\per\mole}
	\end{array}
$$	

\paragraph{Reformage secondaire}

\begin{chemmath}
		CH_4 + \frac{1}{2}O_2 \Longrightarrow CO + 2H_2
\end{chemmath}

$$
	\begin{array}{rl}
	\Delta H_{reaction}		&= \Sigma \Delta H_{f,produits} - \Sigma \Delta H_{f,reactifs}\\
												&= (-110.53) + 2\cdot 0 - ((-74.81) + \frac{1}{2}\cdot 0)\\
												&=  (-110.53) - (-74.81)\\
												&=  -\unit{35.72}{\kilo\joule\per\mole}
	\end{array}
$$

Le reformage secondaire s'opere generalement a une temperature de $\unit{1173}{\kelvin}$.
Voici les $C_p$ variables en fonction de la température des différents composants \cite{hc-table} 
(exprimées en \unit{\joule\per\mole\kelvin}) :

$$
	\left\{
		\begin{array}{rl}
			C_{p_{CH_4}}(T) 	&= 14.23 + 75.3\cdot10^{-3}T + (-18\cdot10^{-6})T^2 \\
			C_{p_{O_2}}(T) 		&= 25.73 + 12.97\cdot10^{-3}T + (-3.77\cdot10^{-6})T^2 \\
			C_{p_{CO}}(T) 		&= 27.62 + 5.02\cdot10^{-3}T + (0\cdot10^{-6})T^2 \\
			C_{p_{H_2}}(T) 		&= 29.3 + (-0.84)\cdot10^{-3}T + (2.09\cdot10^{-6})T^2
		\end{array}
	\right.
$$

Calculons maintenant le $\Delta H$ a $\unit{1173.15}{\kelvin}$ :

$$
	\begin{array}{rl}
		 	\Delta H(\unit{1173.15}{\kelvin}) &=  \Delta H(\unit{298.15}{\kelvin}) 
																						+ \int_{1173.15}^{298.15} C_{p_{reactifs}} dT + \int_{298.15}^{1173.15} 
																						C_{p_{produits}} dT \\
																				&=  \unit{-20.29}{\kilo\joule}
	\end{array}
$$	

Il s'agit donc d'une réaction exothermique.

\paragraph{Water-Gas-Shift}

\begin{chemmath}
		CO + H_2O \Longrightarrow H_2 + CO_2
\end{chemmath}	

$$
	\begin{array}{rl}
	\Delta H_{reaction}		&= \Sigma \Delta H_{f,produits} - \Sigma \Delta H_{f,reactifs} \\
												&= 0 + (-393.51) - (-110.53 + (-241.82)) \\
												&= -393.51 - (-352.35) \\
												&= \unit{-41.16}{\kilo\joule\per\mole}
	\end{array}
$$

Le Water Gas Shift s'opere generalement entre 200 et \unit{400}{\degreecelsius}. Nous considérerons 
alors une température de réaction de $\unit{300}{\degreecelsius}$, soit $\unit{573.15}{\kelvin}$.
						
Voici les $C_p$ variables en fonction de la température des différents composants \cite{hc-table}
(exprimées en \unit{\joule\per\mole\kelvin}) :

$$
	\left\{
		\begin{array}{rl}
			C_{p_{CO_{2}}}(T) &= 32.22 + 22.18\cdot10^{-3}T + (-3.35\cdot10^{-6})T^2\\
			C_{p_{H_2O}}(T)		&= 30.13 + 10.46\cdot10^{-3}T \\
			C_{p_{CO}}(T) 		&= 27.62 + 5.02\cdot10^{-3}T \\
			C_{p_{H_2}}(T) 		&= 29.3 + (-0.84)\cdot10^{-3}T + (2.09\cdot10^{-6})T^2
		\end{array}
	\right.
$$
					
Calculons maintenant le $\Delta H$ à $\unit{573.15}{\kelvin}$ :			

$$
	\begin{array}{rl}
		 	 \Delta H(\unit{573.15}{\kelvin})	&=  \Delta H(\unit{298.15}{\kelvin}) 
				+ \int_{573.15}^{298.15} C_{p_{reactifs}} dT + \int_{298.15}^{573.15} C_{p_{produits}} dT \\
				&=  \unit{-55.63}{\kilo\joule}
	\end{array}
$$	
	
Il s'agit donc d'une réaction exothermique.

\paragraph{Séparation de $CO_{2}$ et de $H_{2}O$}		
Pour cette étape, nous ferons l'hypothèse que les étapes requises pour enlever le $CO_{2}$ et le $H_{2}O$ 
des composants présents dans le circuit ne nécessitent pas d'énergie, ou du moins sont énergétiquement indépendantes 
des autres besoins en énergie du reste du système.

\paragraph{Synthèse du $NH_{3}$} 
\begin{chemmath}
		\frac{1}{2}N_{2} + \frac{3}{2}H_2 \Longrightarrow NH_3 
\end{chemmath}	

$$
	\begin{array}{rl}
	\Delta H_{reaction}		&= \Sigma \Delta H_{f,produits} - \Sigma \Delta H_{f,reactifs} \\
												&= \unit{-46}{\kilo\joule\per\mole}
	\end{array}
$$

La réaction se passe à $\unit{750}{\kelvin}$.
						
Voici les $C_p$ variables en fonction de la température des différents composants \cite{hc-table} :

$$
	\left\{
		\begin{array}{rl}
			C_{p_{N_2}}(T) 	&= 27.62 + 4.19\cdot10^{-3}T \\
			C_{p_{H_2}}(T)	&= 29.3 - 0.84\cdot10^{-3}T + (2.09\cdot10^{-6})T^2\\
			C_{p_{NH_3}}(T) 	&= 31.81 + 15.48\cdot10^{-3}T + (5.86\cdot10^{-6})T^2 \\
		\end{array}
	\right.
$$
					
Calculons maintenant le $\Delta H$ à $\unit{750}{\kelvin}$ :			
$$
	\begin{array}{rl}
		 	 \Delta H(\unit{750}{\kelvin})	&=  \Delta H(\unit{298.15}{\kelvin}) 
																							+ \int_{750}^{298.15} C_{p_{reactifs}} dT + \int_{298.15}^{750} C_{p_{produits}} dT \\
																				&=  \unit{-27.98}{\kilo\joule}
	\end{array}
$$	
	
Il s'agit donc d'une réaction exothermique.

% TODO Ici il y a des trucs à corriger, à voir avec les commentaires de la tutrice.
\paragraph{Conclusion}
En fixant la température du reformage primaire à \unit{1000}{\kelvin}, nous pouvons déterminer les enthalpies des deux 
réactions qui s’y passent :
\begin{itemize}
	\item $\Delta H1 =$ \unit{225.99}{\kilo\joule} par mole de \chemform{CH_4} qui réagit, par seconde ;
	\item $\Delta H2 =$ \unit{-36.25}{\kilo\joule} par mole de \chemform{CO} qui réagit, par seconde.
\end{itemize}

La quantité de chaleur dont nous avons besoin lors du reformage primaire peut alors être déterminée selon l’équation suivante : 
$$Q = \Delta H1\cdot n_{\ce{CH4}} +\Delta H2\cdot n_{\ce{CO}} = 225.99\cdot n_{\ce{CH4}} - 36.250\cdot n_{\ce{CO}}$$
(où $Q$ est en kJ/s) 

Le four ne transmettant que 75\% de sa chaleur au reformage primaire, il nous faut alors multiplier la chaleur 
obtenue par un facteur 4/3 afin de connaître la quantité de chaleur réelle que devrait émettre la réaction de combustion 
du CH4 dans le four.

% ------------------------------------------
% CALCUL DU NOMBRE DE TUYAUX
% ------------------------------------------

\section{Calcul du nombre de tubes}
Nous allons maintenant calculer le nombre de tubes dont nous aurons besoin
pour notre réacteur multi-tubulaire. Ces tubes, d'un rayon $r = \unit{5}{\centi\meter}$,
doivent permettrent le passage des réactifs à l'entrée du reformage primaire avec une
vitesse superficielle $c = \unit{2}{\meter\per\second}$. 
Nous expliquerons la modélisation dans un premier temps, et prendrons un exemple ensuite.

Nous pouvons déterminer le débit volumique, noté $\dot{V}$, grâce à l'expression de la loi des gaz
parfaits $p\dot{V} = \dot{n}RT$ où $\dot{n}$ est le débit molaire, $R$ est la constante des gaz parfaits,
$T$ est la température imposée dans le reformage primaire et $p$ est la
pression dans le réformage primaire, c'est à dire 31 bars :

$$\dot{V} = \frac{\dot{n}RT}{p} = \frac{\dot{n}\cdot 8.314\cdot T}{31\cdot10^5}
 = (2.68\cdot10^{-6})\cdot \dot{n}T$$

Ensuite, à l'aide des notions de système ouvert et de l'hypothèse 
$\dot{m_{\text{entrée}}} = \dot{m_{\text{sortie}}}$, nous obtenons que $\dot{V} 
= c \cdot  A $ avec $ \dot{V}$ où $A$ est la somme des sections de tous les tubes.
En remplaçant par les valeurs que nous possédons, nous obtenons :

$$A = \frac{\dot{V}}{c} = \frac{(2.68\cdot10^{-6})\cdot \dot{n}T}{2} 
= \unit{(1.34\cdot10^{-6})\cdot \dot{n}T}{\meter\squared}$$

Or, on sait que la surface d'un tube vaut $\pi r^2
= \unit{7.85\cdot10^{-3}}{\meter\squared}$. Le nombre de tubes est,
dès lors, le rapport de la section totale $A$ trouvée plus haut sur
la section d'un tube. Ce qui nous donne finalement : 

$$\text{Nombre de tubes} = \frac{(1.34\cdot10^{-6})\cdot \dot{n}T}{7.854\cdot10^{-3}}
= (1.707\cdot10{-4})\cdot\dot{n}T$$

Pour terminer cette partie, nous allons calculer le nombre de tubes
nécessaires pour la production de \unit{1500}{\ton\per\dday} de \chemform{NH_3}
à une température $T = \unit{1080}{\kelvin}$. L'outil de gestion
nous donne alors $\dot{n}_{\text{réactifs}} = \dot{n}_{CH_4} + \dot{n}_{H_2O} 
= \unit{872.22}{\mole\per\second}$.

On trouve alors immédiatement :

$$\text{Nombre de tubes} = 160.83$$

que l'on arrondit bien sur à l'unité supérieure pour obtenir $161$.

% ------------------------------------------
% OUTIL DE GESTION
% ------------------------------------------
\section{Outil de gestion}
Notre outil de gestion se base sur les équations écrites lors du
bilan de matière et lors du calcul de l'état d'avancement des réactions
dans le reformage primaire. Il résout donc un système à 4 équations
et à 4 inconnues et ne sélectionne que les solutions positives et réelles.
Il présente les résultats sous forme d'un tableau reprenant les flux de chaque
composant à chaque étape du procédé. Les résultats sont présentés en tonnes par jour
et en moles par seconde.
Nous avons également intégré le calcul du nombre de tubes nécessaires au
passage du mélange \chemform{CH_4(g)} et \chemform{H_2O(g)}.

Notre outil de gestion utilise deux autres fonctions \lstinline{ComputeK1}
et \lstinline{ComputeK2} qui permettent respectivement de calculer la constante
d'équilibre et la variation d'enthalpie de la première et de la deuxième
réaction du reformage primaire pour une température $T$.
Une interface graphique est mise à votre disposition pour une utilisation plus aisée de notre outil de gestion.
Pour l'utiliser, il faut lancer le fichier \textbf{OutilDeGestionGraphicalUserInterface.m}
(attention de ne pas le confondre avec \textit{OutilDeGestionGraphicalUserInterface.fig},
avec lequel les calculs \textbf{ne fonctionneront pas} si on l'ouvre directement) et appuyer sur "Run" dans \textsc{MATLAB}.
Il vous faudra ensuite entrer le débit de \chemform{NH_3} requis et la température du réacteur du réformage primaire. 
Vous avez le choix entre une méthode considérant la dernière réaction (procédé Haber-Bosch) comme
complète et une méthode prenant en compte un recyclage des réactifs avec une purge de 4\% au niveau de ce même procédé.   

% ------------------------------------------
% ETUDE PARAMETRIQUE
% ------------------------------------------
\section{Etude paramétrique}
Pour l'étude paramétrique, nous avons fait des
graphes en fonction de la température de chaque
débit intermédiaire, tout ces graphes (exceptés
les graphes constants, qui n'ont pas beaucoup
d'intèrêt) sont présents dans l'annexe 
\ref{sec:graphes}. Comme la température
standard du reformage primaire est de 
\unit{1080}{\kelvin}, nous avons décidé
de faire des graphes pour une température
comprise entre 800 et \unit{1200}{\kelvin}.
De même, comme la production d'ammoniac
est comprise entre 1000 et \unit{2000}{\ton\per\dday},
nous avons décidé de faire ces graphes pour
une production intermédiaire de 
\unit{1500}{\ton\per\dday}.

\paragraph{Domaine de faisabilité}
A partir de ces graphes nous pouvons
établir un domaine de faisabilité de
la réaction. En observant le graphe
du débit molaire de \ce{H2O} (voir 
figure \ref{fig:etude-h2o}, à droite) entrant
(et sortant) de la phase de séparation,
on observe que ce débit molaire
devient négatif pour une température
approximative de \unit{1035}{\kelvin}.
Cette situation n'est bien sur pas
possible et elle constitue donc une
limite de notre procédé.

\biblio{sources-tache1}

\newpage
\annexe
\subsection{Premier jet du flow-sheet simplifié}
\label{app:flow-sheet}
La première ébauche de notre flow-sheet se trouve à la figure \ref{flow-sheet}.

\begin{figure}[htb!]
	\centering
	\rotatebox{90}{\includegraphics[scale=0.40]{media/flow-sheet.png}}
	\caption{Première ébauche de notre flow-sheet.}
	\label{flow-sheet}
\end{figure}
\newpage

\subsection{Deuxième version du flow-sheet}
\label{appendix:flow-sheet}
La deuxième version de notre flow-sheet se trouve à la figure \ref{flow-sheet-v2}.

\begin{figure}[htb!]
	\centering
	\includegraphics[scale=0.65]{media/flow-sheet-v2.jpg}
	\caption{Deuxième ébauche de notre flow-sheet.}
	\label{flow-sheet-v2}
\end{figure}
\newpage

% Useless
%\section{Code Matlab de l'outil de gestion}
%\label{code-matlab-outil}
%\lstinputlisting{matlab/OutilDeGestionV2.m}
%\lstinputlisting{matlab/ComputeK1.m}
%\lstinputlisting{matlab/ComputeK2.m}

\subsection{Graphes des débits intermédiaires en fonction
de la température}
\label{sec:graphes}

\begin{figure}[htb!]
	\centering
	\includegraphics[scale=0.70]{media/etude_param/useful/tubes.jpg}
\end{figure}

\begin{figure}[htb!]
	\centering
	\subfigure{
		\includegraphics[scale=0.40]{media/etude_param/useful/ch4_in2_four.jpg}
	}
	\subfigure{
		\includegraphics[scale=0.40]{media/etude_param/useful/o2_in1_four.jpg}	
	}
\end{figure}

\begin{figure}[htb!]
	\centering
	\subfigure{
		\includegraphics[scale=0.40]{media/etude_param/useful/ch4_in1_reformage_primaire.jpg}
	}
	\subfigure{
		\includegraphics[scale=0.40]{media/etude_param/useful/h2o_in1_reformage_primaire.jpg}
	}
\end{figure}

\begin{figure}[htb!]
	\centering
	\subfigure{
		\includegraphics[scale=0.40]{media/etude_param/useful/ch4_in3_reformage_secondaire.jpg}
	}
	\subfigure{
		\includegraphics[scale=0.40]{media/etude_param/useful/co_in1_reformer_secondaire.jpg}
	}
\end{figure}

\begin{figure}[htb!]
	\centering
	\subfigure{
		\includegraphics[scale=0.40]{media/etude_param/useful/co2_in1_reformer_secondaire.jpg}
	}
	\subfigure{
		\includegraphics[scale=0.40]{media/etude_param/useful/h2_in1_reformer_secondaire.jpg}	
	}
\end{figure}

\begin{figure}[htb!]
	\centering
	\includegraphics[scale=0.70]{media/etude_param/useful/h2o_in2_reformage_secondaire.jpg}
\end{figure}

\begin{figure}[htb!]
	\centering
	\subfigure{
		\includegraphics[scale=0.40]{media/etude_param/useful/co_in2_wgs.jpg}
	}
	\subfigure{
		\includegraphics[scale=0.40]{media/etude_param/useful/co2_in2_wgs.jpg}	
	}
\end{figure}

\begin{figure}[htb!]
	\centering
	\subfigure{
		\includegraphics[scale=0.40]{media/etude_param/useful/h2_in2_wgs.jpg}
	}
	\subfigure{
		\includegraphics[scale=0.40]{media/etude_param/useful/h2o_in3_wgs.jpg}	
	}
\end{figure}

\begin{figure}[htb!]
	\centering
	\subfigure{
		\includegraphics[scale=0.40]{media/etude_param/useful/co2_in_out_separation.jpg}
	}
	\subfigure{
		\includegraphics[scale=0.40]{media/etude_param/useful/h2o_in_out_separation.jpg}	
	}
	\caption{Sur la figure de droite, on peut remarquer que le débit d'eau devient
	négatif pour une température d'appriximativement \unit{1035}{\kelvin}.}
	\label{fig:etude-h2o}
\end{figure}

\input{../footer.tex}
