\documentclass{beamer} 

\usepackage[utf8]{inputenc} 
\usepackage[frenchb]{babel}
\usepackage[T1]{fontenc}
\usepackage{lmodern}
\usepackage{graphicx} 
\usepackage[squaren, Gray]{SIunits} 
\usepackage{amsmath}  
\usepackage{tabularx} 
\usepackage{chemist} 
\usepackage[version=3]{mhchem}
\usepackage{color}

\usetheme{warsaw} 
 
\title{Projet P3} 
\subtitle{Introduction au génie chimique : analyse du procédé de production d'ammoniac} 
\author{\textbf{Groupe 124.3}\\
\textsc{Frenyo} Péter (6266-12-00)\\
\textsc{Gillain} Nathan (7879-12-00)\\
\textsc{Lamine} Guillaume (7109-13-00)\\
\textsc{Piraux} Pauline (2520-13-00)\\
\textsc{Paris} Antoine (3158-13-00)\\
\textsc{Quiriny} Simon (4235-13-00)\\
\textsc{Schrurs} Sébastien (7978-13-00)}
\date{\today}
 
\begin{document} 
 
	\begin{frame} 
		\titlepage 
	\end{frame} 
	
	\begin{frame}
		\frametitle{Plan de l'exposé}
		\tableofcontents
	\end{frame}
	
	\begin{frame}
		\section{Introduction}
		\frametitle{Plan de l'exposé}
		\tableofcontents[currentsubsection,sectionstyle=show/shaded,subsectionstyle=show/shaded/hide]
	\end{frame}
	
	\begin{frame}
		\frametitle{Plan de l'exposé}
		\section{Tâche 3 - analyse de l'impact environnemental}
		\tableofcontents[currentsubsection,sectionstyle=show/shaded,subsectionstyle=show/shaded/hide]
	\end{frame}
	
	\begin{frame}
	\frametitle{Analyse de l'impact environnemental : Démarche}
	\begin{itemize}
		\item[-] Recherche des valeurs à quantifier grâce à un brainstorming ;
		\item[-] Recherche des différentes températures des réacteurs ;
		\item[-] Quantification des flux de produits secondaires grâce à l'outil de gestion ;
		\item[-] Calcul de l'énergie dégagée/absorbée par les différentes réactions ;
		\item[-] Pistes d'amélioration.
	\end{itemize}
	\end{frame}

	\begin{frame}
	\frametitle{Analyse de l'impact environnemental : Résultats}
	Pour une production de \unit{1500}{\ton\per\dday} avec une température 
	de \unit{1000}{\kelvin} dans le reformage primaire, nous 	produisons pour tout le procédé :
	\begin{itemize}
		\item[-] \unit{1725}{\ton\per\dday} de \chemform{CO_2} ;
		\item[-] Entre 0.9 et \unit{1.95}{\ton\per\dday} de \chemform{NO_x} ;
		\item[-] \unit{-53.75}{\kilo\joule\per\dday} ;
		\item[-] \unit{22.6}{\ton\per\dday} de \chemform{Ar}.
	\end{itemize}
	\end{frame}

	\begin{frame}
	\frametitle{Analyse de l'impact environnemental : Pistes pour améliorer le procédé}
	Nous avons réfléchi aux divers points négatifs et avons trouvé quelques pistes pour y remédier :
	\begin{itemize}
		\item[-] Utiliser un procédé de production de dihydrogène moins polluant(électrolyse, partial oxydation, ...) .
		\item[-] Chauffer le reformage primaire avec une source d'énergie verte ;
		\item[-] Récupérer l'énergie dégagée par les diverses réactions exothermiques ;
		\item[-] Reconvertir le \chemform{CO_2} et les autres déchets produits ou les vendre ;
		\item[-] Utiliser d'autres matières premières pour la production de dihydrogène et ;
		de diazote et éviter les poisons catalytiques à traiter.
	\end{itemize}
	\end{frame}
	
	\begin{frame}
		\section{Tâche 8 - amélioration du procédé}
		\frametitle{Plan de l'exposé}
		\tableofcontents[currentsubsection,sectionstyle=show/shaded,subsectionstyle=show/shaded/hide]
	\end{frame}
	
	\begin{frame}
		\frametitle{Démarche}
		\framesubtitle{Analyse des enjeux environnementaux}
		% TODO
	\end{frame}
	
	\begin{frame}
		\frametitle{Démarche}
		\framesubtitle{Choix d'une source d'impact et pistes d'amélioration} % FIX : s ou pas?
		Notre choix : le \chemform{CO_2}.\\
		Deux possibilités : soit \textbf{réduire les émissions}, soit \textbf{recycler}.
		
		Pour reduire les émissions :
		\begin{itemize}
			\item Changer le procédé de combustion ;
			\item Changer le procédé de création de dihydrogène.
		\end{itemize}
		Pour recycler :
		\begin{itemize}
			\item Produire du carburant à partir d'algues ;
			\item Recycler en matière première ;
			\item Revendre le \chemform{CO_2} à d'autres usines
			en ayant besoin.
		\end{itemize}
	\end{frame}
	
		\begin{frame}
		\frametitle{Démarche}
		\framesubtitle{Choix d'une source d'impact et pistes d'amélioration} % FIX : s ou pas?
		Notre choix : le \chemform{CO_2}.\\
		Deux possibilités : soit \textbf{réduire les émissions}, soit \textbf{recycler}.
		
		Pour reduire les émissions :
		\begin{itemize}
			\item Changer le procédé de combustion ;
			\item Changer le procédé de création de dihydrogène.
		\end{itemize}
		Pour recycler :
		\begin{itemize}
			\item \textbf{Produire du carburant à partir d'algues} ;
			\item Recycler en matière première ;
			\item Revendre le \chemform{CO_2} à d'autres usines
			en ayant besoin.
		\end{itemize}
	\end{frame}
	
	\begin{frame}
		\frametitle{Notre proposition : l'algocarburant}
		\framesubtitle{Fonctionnement}
		\begin{figure}
			\centering
			\includegraphics[scale=0.6]{media/fonctionnement.png}
		\end{figure}
	\end{frame}
	
	\begin{frame}
		\frametitle{Notre proposition : l'algocarburant}
		\framesubtitle{Facteurs importants pour le développement des micro-algues}
		\Large{\begin{itemize}
			\item Luminosité (rayons UV) ;
			\item Température ;
			\item Régulation des nutriments ;
			\item Qualité du \chemform{CO_2} ;
			\item Espèce d'algue.
		\end{itemize}}
	\end{frame}
	
	\begin{frame}
		\frametitle{Nos arguments}
		\framesubtitle{Avantages...}
		\begin{center}
			\small{
			\begin{tabular}{p{0.45\textwidth}|p{0.45\textwidth}}
				\textbf{Micro-algues} & \textbf{Algocarburants} \\
				\hline
					\begin{itemize}
						\item[\textcolor{green}{+}] Croissance ;
						\item[\textcolor{green}{+}] Pas de compétition avec les cultures alimentaires;
						\item[\textcolor{green}{+}] Rendement ;
						\item[\textcolor{green}{+}] Faible emprunte environnementale ;
						\item[\textcolor{green}{+}] Facilité à cultiver.
					\end{itemize}   & 
					\begin{itemize}
						\item[\textcolor{green}{+}] Directement consommable par nos moteurs ;
						\item[\textcolor{green}{+}] Rejets de \chemform{CO_2} moins élevés.
					\end{itemize}
			\end{tabular}}
		\end{center}
	\end{frame}
	
	\begin{frame}
		\frametitle{Nos arguments}
		\framesubtitle{... mais aussi quelques inconvénients}
		\begin{itemize}
				\item[\textcolor{red}{-}] Faute de production en masse : prix élevés ;
				\item[\textcolor{red}{-}] Extraction de l'huile coûteuse et dévoreuse d'énergie ;
				\item[\textcolor{red}{-}] Nécessité de rendre le \chemform{CO_2} propre à la 
				consommation des algues ;
				\item[\textcolor{red}{-}] Quantité élevé d'azote et de phosphore élevé
				dans la biomasse.
		\end{itemize}
	\end{frame}
	
	\begin{frame}
		\frametitle{Nos arguments}
		\framesubtitle{Etude quantitative}
		Notre production de \chemform{CO_2} :
		\begin{itemize}
			\item Procédé : \unit{x}{\ton} par an ;
			\item Combustion : \unit{x}{\ton} par an.
		\end{itemize}
		Production des micro-algues :
		\begin{itemize} 
			\item \unit{1}{\hectare} d'algue $\approx$ \unit{x}{\kilo\gram} de biomasse $\approx$
			\unit{y}{\kilo\gram} d'huile $\approx$ \unit{z}{\liter} de carburant ;
			\item \unit{1}{\kilo\gram} de biomasse $\approx$ \unit{1.8}{\kilo\gram}
			de \chemform{CO_2} fixé.
		\end{itemize}
		\fbox{\begin{minipage}{0.9\textwidth}Avec \unit{x}{\hectare} d'algues, on produit \unit{x}{\liter}
		de carburant et on recycle \unit{x}{\ton} de \chemform{CO_2} par
		an. C'est à dire X \% de nos émissions.\end{minipage}}
	\end{frame}

	\begin{frame}
		\frametitle{Nos arguments}
		\framesubtitle{D'un point de vue économique}
		
	\end{frame}

	\begin{frame}
		\section{Conclusion des tâches 3 et 8}
		\frametitle{Plan de l'exposé}
		\tableofcontents[currentsubsection,sectionstyle=show/shaded,subsectionstyle=show/shaded/hide]
	\end{frame}
	
	\begin{frame}
		\section{Bilan de groupe}
		\frametitle{Plan de l'exposé}
		\tableofcontents[currentsubsection,sectionstyle=show/shaded,subsectionstyle=show/shaded/hide]
	\end{frame}
	
	\begin{frame}
		% TODO
	\end{frame}
	
	\begin{frame}
		\section{Conclusion du projet}
		\frametitle{Plan de l'exposé}
		\tableofcontents[currentsubsection,sectionstyle=show/shaded,subsectionstyle=show/shaded/hide]
	\end{frame}
	
	\begin{frame}
		% TODO
	\end{frame}
	
\end{document}