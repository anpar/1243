\documentclass{article}
\usepackage[numbered, framed]{mcode}
% Langue
\usepackage[utf8]{inputenc}
\usepackage[T1]{fontenc}      
\usepackage[francais]{babel}

% Mise en forme générale
\usepackage[top=2.5cm,bottom=2.5cm,right=2.5cm,left=2.5cm]{geometry}
\usepackage{subfigure}
% Package divers
\usepackage{chemist} 
\usepackage[version=3]{mhchem}
\usepackage{chemfig}
\usepackage[squaren, Gray]{SIunits}
\usepackage{sistyle}
\usepackage[autolanguage]{numprint}
\usepackage{url}
\usepackage{rotating}
\usepackage{xcolor,colortbl}
\definecolor{Gray}{gray}{0.85}

\usepackage{hyperref}
\hypersetup{
    colorlinks,
    citecolor=black,
    filecolor=black,
    linkcolor=black,
    urlcolor=black
}

% Nouvelles commandes
\newcommand{\std}{\ensuremath{^{\circ}}}
\newcommand\ph{\ensuremath{\mathrm{pH}}}
\newcommand{\annexe}{\part{Annexes}\appendix}
\newcommand{\biblio}[1]{\bibliographystyle{plain}\bibliography{#1}\nocite{*}}

\newcommand{\doctitle}[1]{
	\title{#1}
	\author{\textbf{Groupe 124.3}\\
	\textsc{Frenyo} Péter (6266-12-00)\\
	\textsc{Gillain} Nathan (7879-12-00)\\
	\textsc{Lamine} Guillaume (7109-13-00)\\
	\textsc{Piraux} Pauline (2520-13-00)\\
	\textsc{Paris} Antoine (3158-13-00)\\
	\textsc{Quiriny} Simon (4235-13-00)\\
	\textsc{Schrurs} Sébastien (7978-13-00)}
	\date{\today}

	\begin{document}

	\maketitle
	\tableofcontents
}
\doctitle{Tache 4 : Etude HAZOP du noeud autour du réacteur de synthèse d'ammoniac}

\section{Dangers présentés par les substances mises en oeuvre durant la synthèse de l'ammoniac}

\section{Dangers présentés par les substances mises en oeuvre durant la synthèse de l'ammoniac}
\subsection{L'azote}
Premièremet, le diazote utilisé est gardé sous pression. Tout gaz comprimé présente un danger. 
En effet, des rejets de gaz comprimé mal contrôlés dans les réacteurs chimiques peuvent entraîner 
la rupture des cuves, créer des fuites dans l'équipement ou les canalisations ou faire emballer la réaction. 
%CECI EST UNE CITATION EXACTE DU SITE SUIVANT : http://www.cchst.com/oshanswers/chemicals/compressed/compress.html#_1_3
%Partie "Quels sont les dangers des matières sous pression", 3e paragraphe, va ptet le lire pour voir si on le met comme citation exacte. 
%Je me sers encore de ce site par la suite mais plus en citation exacte. 
Si le contenant du gaz n'est de plus pas solidement fixé, cela peut entraîner un effet dit "fusée" 
et causer des dommages et blessures.

Le diazote est également un gaz toxique et peut entraîner des morts par asphyxie dans les espaces confinés. 
%il est nécessaire de vérifier la présence d'une proportion suffisante d'oxygène dans de tels espaces confinés 
%avant d'y pénétrer, ou de s'équiper d'un appareil respiratoire autonome. (Wikipedia) < Voit si tu veux rajouter ça en plus, ou c'est plutôt une solution.
\subsection{L'hydrogène}
Les dihydrogène étant également comprimé, il présente les même dangers de gaz sous pression que mentionnés pour le diazote.

De plus, le dihydrogène est un gaz extrêmement inflammable, réactif et explosif. Un choc, une étincelle ou autre peut facilement 
entraîner une combustion rapide pouvant menant à une explosion.

L'hydrogène peut également corroder certains métaux et être source de fragilités ou fissures sur le matériel, et présente un danger de suffocation par inhalation.
\subsection{L'argon}
L'argon étant également maintenu sous pression, les même dangers que mentionnés pour l'azote sont présents.

L'argon en forte concentration peut réduire la teneur en oxygène du milieu, provoquant des pertes 
de consciences ou, dans le pire des cas, des morts par asphyxies. %Source : même site que pour le diazote, partie "Danger des gaz inertes".

\subsection{L'ammoniac}
L'ammoniac est, encore une fois, maintenu sous pression, donc les dangers des gaz sous pressions sont de nouveau présents ici.

L'ammoniac est également corrosif : son contact peut brûler et détruire les tissus, et peut également attaquer et corroder 
les métaux. Il est classé comme matière "très toxique ayant des effets immédiats graves". Il est irritant et toxique pour 
les êtres vivants et l'environnement. %Source : même site + wikipedia "ammoniac". Osef des wikipedia c'est juste de l'autre 
%site que je me suis méga-inspiré pour quasi chaque partie ici.
\section{Trajectoire du flux}
% TODO Ajouter le scan des feuilles.

\section{Pourquoi n'y a-t-il pas de soupape de sécurité ou de disque de rupture sur
le réacteur de synthèses d'ammoniac?}
Dans le réaceur de synthèse, on a la réaction suivante : 

\begin{chemmath}
  3H_2(g) + N_2(g) \rightarrow 2NH_3(g)
\end{chemmath}

On peut donc voir que pour 4 moles de gaz de réactifs, 2 moles de gaz sont produites.
Puisque le nombre de moles de gaz diminue, la pression aura tendance à diminuer quand la réaction se fait. 
C'est pour cela qu'on ne craint pas la surpression et qu'aucun dispositif n'a été mis en place pour cela.

\section{Pourquoi y a-t-il des disques de rupture sur l'échangeur 124-MC ?}
% TODO Ajouter l'explication de Nathan

\section{Analyse HAZOP}

	\begin{table}[ht!]
		\centering
		\rotatebox{90}
		{
			\begin{tabular}{c|c|c|c}
				\rowcolor{Gray} Mot-guide		& Causes 	& Conséquences 	&	Mesures de maîtrise 	\\
				\hline
				Trop de corrosion		  &  Une "hydrogen attack" due à réaction à haute pression de l'hydrogène avec l'acier. Lieu: Du début jusqu'à la chambre 1 du 105MD	& 	Les tuyaux sont endommagés (percés ou présence du fuites)ce qui peut même mener à une explosion quand l'hydrogène et l'oxygène rentrent en contact. Lieu: Du début jusqu'à la chambre 1 du 105MD.	 &  Contrôle des matériaux	et augmentation de leur qualité. Prévoir les revêtements adéquats pour éviter tout contact entre acier et hydrogène. 								 	\\				
				\hline
				Température	trop basse	&  Liquéfaction/condensation de l'ammoniac juste après le 124MC.	&  Tuyaux bouchés ce qui peut entrainer une surpression juste après le 124MC.  &    Implosion des tuyaux.		 									 	\\
				\hline 
				Trop d'usure, corrosion	& Dégradation des installations avec le temps et impureté des produits dans les conduits. Lieu: Dans toutes les canalisations mais principalement entre le 105MD et le 123MC1 du à la haute pression.	&  Entraîne des réactions indésirées qui amènent des impuretés dans l'ammoniac. Lieu: Dans toutes les canalisations mais principalement entre le 105MD et le 123MC1 du à la haute pression.	 &  Contrôler les installations	tous les ans et mettre un Filtre physique pour avoir de l'ammoniac pur.	 									 	\\
				\hline
				Température trop haute	&	Surpression dans le réacteur de synthèse d'ammoniac (105MD)				& Peut entrainer des fissures dans la paroie voir même la destruction du réacteur. Il y alors risque d'explosion. (105MD)							& Présence d'un disque de rupture pour éviter la surpression		 									 	\\
				\hline
			\end{tabular}
		}
		\caption{Synthèse de l'analyse HAZOP.}
	\end{table}

\end{document}
