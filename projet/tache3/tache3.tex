\documentclass{article}
\usepackage[numbered, framed]{mcode}
% Langue
\usepackage[utf8]{inputenc}
\usepackage[T1]{fontenc}      
\usepackage[francais]{babel}

% Mise en forme générale
\usepackage[top=2.5cm,bottom=2.5cm,right=2.5cm,left=2.5cm]{geometry}
\usepackage{subfigure}
% Package divers
\usepackage{chemist} 
\usepackage[version=3]{mhchem}
\usepackage{chemfig}
\usepackage[squaren, Gray]{SIunits}
\usepackage{sistyle}
\usepackage[autolanguage]{numprint}
\usepackage{url}
\usepackage{rotating}
\usepackage{xcolor,colortbl}
\definecolor{Gray}{gray}{0.85}

\usepackage{hyperref}
\hypersetup{
    colorlinks,
    citecolor=black,
    filecolor=black,
    linkcolor=black,
    urlcolor=black
}

% Nouvelles commandes
\newcommand{\std}{\ensuremath{^{\circ}}}
\newcommand\ph{\ensuremath{\mathrm{pH}}}
\newcommand{\annexe}{\part{Annexes}\appendix}
\newcommand{\biblio}[1]{\bibliographystyle{plain}\bibliography{#1}\nocite{*}}

\newcommand{\doctitle}[1]{
	\title{#1}
	\author{\textbf{Groupe 124.3}\\
	\textsc{Frenyo} Péter (6266-12-00)\\
	\textsc{Gillain} Nathan (7879-12-00)\\
	\textsc{Lamine} Guillaume (7109-13-00)\\
	\textsc{Piraux} Pauline (2520-13-00)\\
	\textsc{Paris} Antoine (3158-13-00)\\
	\textsc{Quiriny} Simon (4235-13-00)\\
	\textsc{Schrurs} Sébastien (7978-13-00)}
	\date{\today}

	\begin{document}

	\maketitle
	\tableofcontents
}
\doctitle{Etude de l'impact environnemental}
\section{Etude de  la consommation énergétique}
Dans cette section, nous allons étudier la 
consommation d'énergie en fonction de la
masse de \chemform{NH_3} et de la température $T$
en Kelvin.
\section{Etude du \chemform{CO_2} rejeté par le four}
% TODO
\section{Etude du rejet total de \chemform{CO_2}}
% TODO
\section{Etude des autres rejets}
Dans cette section, nous allons
évoquer les risques liés aux autres 
élements que rejettent notre procédé
de fabrication, à savoir l'argon et
le diazote. Nous ne parlerons
évidemment pas du cas de l'eau 
rejetée car celle-ci n'a aucun impact
sur l'environnement.
\subsection{L'argon}
En faible quantité, l'argon n'est
pas dangereux pour la santé (puisqu'il
entre dans la composition de l'air...).
L'argon ne présente également aucun risque
pour l'environnement\footnote{Ou en tout
cas aucun de ces risques ne sont connus
à l'heure actuel.}, ni pour les plantes, 
ni pour les animaux, ni même pour le monde
aquatique.
% http://www.lenntech.fr/data-perio/ar.htm
\subsection{Azote}
Contrairement à l'argon, le rejet 
d'azote est mauvais pour l'environnement.
Les conséquences sont multiples :
\begin{itemize}
	\item Modifications des concentrations
	des substances sensibles à l'azote
	prénsentes dans l'air ;
	\item Dangés pour la santé du bétail
	environnant (et donc pour les consommateurs).
\end{itemize}
% http://www.lenntech.fr/data-perio/n.htm

\section{Recommandations pour réduire
l'impact environnemental}
% TODO
\biblio{sources-tache3}
\input{../footer.tex}